\documentclass{supervision} 

\begin{document}
	
\newcommand{\course}{Concepts in Programming Languages}
\newcommand{\week}{I}
\newcommand{\topics}{}

\begin{center}
\LARGE {\textbf{\color{campurpledark} \course} }\\[-0.2cm]
\Large \color{campurpledark} Supervision \week: The Ancestors\\
\end{center}

{\color{campurple}\hrule}

\newcommand{\terminal}[1]{\texttt{\color{campurple}#1}}
\newcommand{\bl}[1]{{\color{black}#1}}

\vspace{0.5cm}

This supervision covers Part A of the lectures. You will have to do some preparatory reading (slides or textbooks) to complete the exercises below. You should prepare notes in response to each of the exercises and bring them with you to the supervision.

\emph{By the end of this supervision}: you should be familiar with the features and innovations of the programming languages from Part A of the course (``The Ancestors''). You should also understand what factors influence the design of a programming language. 

\begin{questions}

\question Answer all parts of this question for each of the following programming languages: FORTRAN, LISP, Algol, Pascal, SIMULA, Smalltalk.
\begin{parts}
\part Give a brief summary of the design of the language.
\part What was the motivation for the language?
\part What are the main innovations of the language?
\end{parts} 
\question In your opinion, what makes a programming language object-oriented? 
\question For your favourite programming language, discuss the following points:
\begin{parts}
	\part What was the motivation for the language?
    \part Does the language have a simple and unambiguous execution model?
    \part Is the language design grounded in theory?
    \part Why is it your favourite?
\end{parts}
\question For your least favourite programming language, discuss the following points:
\begin{parts}
	\part What was the motivation for the language?
    \part Does the language have a simple and unambiguous execution model?
    \part Is the language design grounded in theory?
    \part Why is it your least favourite?
\end{parts}
\question Discuss what factors you consider important for successful language design.
\question If you were to design a programming language to suit your personal preferences and needs, what would that language look like?
\end{questions}
\end{document}
